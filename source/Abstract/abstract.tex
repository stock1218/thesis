\abstract

The first major integer safety issue occurred in 1975 with the DATE75 bug, where dates past January 4th 1975 could not be represented using a 12 bit integer. Almost 50 years later, integer safety bugs are still an issue programmers must consider when writing code. The root cause of this issue is related to how computers represent numbers. That is, numbers are represented in binary using a fixed number of bits. In C, when a value is placed in some variable that does not have enough bits to store said value, integer overflow or wraparound occurs. This is essentially a truncation operation and reflects how the underlying hardware operates. Integer overflow is when a signed integer is too small, and wraparound is when an unsigned integer is too small. Although integer wraparound is defined by the C standard, both types of behavior can result in unexpected values and develop into security vulnerabilities. Recently, the C standards committee has come out with a new set of checked arithmetic macros for combating these issues. C23 now has checked arithmetic macros that can be used to ensure that integer operations do not overflow or unintentionally wraparound. This thesis aims to explore the application of these macros, and propose a solution to automatically updating older codebases using a compiler-assisted approach.

There are a number of issues related to using the new checked arithmetic macros, and even more so trying to automatically rewrite expressions. When a rewrite needs to be performed, a set of temporary variables need to be defined based on the types involved in the original expression. These types may be non-obvious, since they may be declared elsewhere or implicitly cast at some point. To address this issue, type information is collected from the compiler to ensure that these temporary variables reflect the types in the statement itself and its destination. This approach was implemented using the tools and APIs available in the LLVM compiler infrastructure project, and was implemented as a clang-tidy check. Using these tools allowed the clang-tidy check to match on statements that could be rewritten, then suggest a fix that would select the intermediate types based on what the compiler was using.

To evaluate the effectiveness of the clang-tidy plugin, several research questions were posed. This included how comprehensive was the rewrite, what were common reasons to omit lines, and what was the performance overhead. The methodology included writing ``debug" checks to match on all possible statements that could overflow to then calculate the proportion of those statements that could be rewritten. Figuring out common reasons to omit lines was purely qualitative and was based on the experience rewriting several targets. Finally, performance overhead was calculated by running tests before and after the rewrite and timing their runtime. Three open-source targets were selected based on their relative size, complexity, and testing infrastructure. The results of the evaluation were coverage ranging from 87\% to 53\% of statements being rewritten. The experience of rewriting was simple for the first two targets, and unsuccessful for the third due to several issues. Finally, of the two targets that could be tested for performance, performance overhead for checked arithmetic came out to be 12\% and 4\% respectively.

In all, this thesis explored the possibility of compiler-assisted code modernization and produced a proof-of-concept clang-tidy plugin that can successfully rewrite targets to use new C23 checked arithmetic macros.